\documentclass[english]{lni}

\IfFileExists{latin1.sty}{\usepackage{latin1}}{\usepackage{isolatin1}}  % comment out this line for compilation with e.g. overleaf

\usepackage{graphicx}
\usepackage{fancyhdr}
\usepackage{listings} %if lstlistings is used
\usepackage{changepage} %for changing topmargin on first page
\usepackage[figurename=Fig., tablename=Tab., small]{caption}[2008/04/01]
\renewcommand{\lstlistingname}{List.}    % Listingname is now List. 

\fancypagestyle{titlepage}{
\fancyhead[RO]{\small N. Damer,  M. Gomez-Barrero,  K. Raja,  C. Rathgeb,  A.  Sequeira,\linebreak  M. Todisco,  and A. Uhl (Eds.): BIOSIG 2023, \linebreak Lecture Notes in Informatics (LNI), Gesellschaft f\"ur Informatik, Bonn 2023} % do NOT modify these lines
\fancyfoot{}}

%Beginning of page count for this paper
\setcounter{page}{1}

%head line settings
\pagestyle{fancy}
\fancyhead{} % clears the settings
\fancyfoot{} % clears footer settings
\renewcommand{\headrulewidth}{0.4pt} %horizontal line below header
\setcounter{footnote}{0}

\author{Gustav Moesmand\footnote{DTU, Compute, Address1, Place1, s174169@dtu.dk}}
\title{Face Beauty Score Implementation (FBI)}
\begin{document}
\maketitle
% Abstract
\renewcommand{\refname}{References}
\setcounter{footnote}{2} %Change to the number of authors for a correct numbering of the foot notes
\thispagestyle{titlepage}
%header setting after the second page
\pagestyle{fancy}
\fancyhead{} % clears header settings
\fancyhead[RO]{\small Short title of your work \hspace{25pt}  \hspace{0.05cm}}
\fancyhead[LE]{\hspace{0.05cm}\small  \hspace{25pt} Firstname1 Lastname1 and Firstname2 Lastname2}
\fancyfoot{} % clears all footer settings
\renewcommand{\headrulewidth}{0.4pt} %line below header

\begin{abstract}
	The \LaTeX-Class \texttt{lni} uses the layout for articles in LNI. This document describes the usage of the \LaTeX-Class and gives some examples. This abstract should give a short overview about your work and can have between 70 and 150 words. The formatting will be done automatically within the abstract area.
\end{abstract}
\begin{keywords}
	LNI Guidelines, \LaTeX Template. The formatting will be done automatically within the keywords-Range.
\end{keywords}

\section{Introduction}
this is a test of how the report looks

\section{Related Work}

\section{Main body}
\subsection{Background}
\subsection{Methodology}
\subsection{Litterature Survey}
\subsection{Experimental evaluation}
\subsubsection{Results}
\subsubsection{Discussion}

\section{Conclusion}



\end{document}
