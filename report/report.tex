\documentclass[english]{lni}

\IfFileExists{latin1.sty}{\usepackage{latin1}}{\usepackage{isolatin1}}  % comment out this line for compilation with e.g. overleaf

\usepackage{graphicx}
\usepackage{fancyhdr}
\usepackage{listings} %if lstlistings is used
\usepackage{changepage} %for changing topmargin on first page
\usepackage[figurename=Fig., tablename=Tab., small]{caption}[2008/04/01]
\renewcommand{\lstlistingname}{List.}    % Listingname is now List. 

\fancypagestyle{titlepage}{
\fancyhead[RO]{} % do NOT modify these lines
\fancyfoot{}}

%Beginning of page count for this paper
\setcounter{page}{1}

%head line settings
\pagestyle{fancy}
\fancyhead{} % clears the settings
\fancyfoot{} % clears footer settings
\renewcommand{\headrulewidth}{0.4pt} %horizontal line below header
\setcounter{footnote}{0}

\author{Gustav Moesmand\footnote{DTU, Compute, s174169@dtu.dk}}
\title{Face Beauty Score Implementation (FBI)}
\begin{document}
\maketitle
\cite[test]{j_iyer_machine_2021}

% Abstract
\renewcommand{\refname}{References}
\setcounter{footnote}{2} %Change to the number of authors for a correct numbering of the foot notes
\thispagestyle{titlepage}
%header setting after the second page
\pagestyle{fancy}
\fancyhead{} % clears header settings
\fancyhead[RO]{\small Short title of your work \hspace{25pt}  \hspace{0.05cm}}
\fancyhead[LE]{\hspace{0.05cm}\small  \hspace{25pt} Firstname1 Lastname1 and Firstname2 Lastname2}
\fancyfoot{} % clears all footer settings
\renewcommand{\headrulewidth}{0.4pt} %line below header

\begin{abstract}
	The \LaTeX-Class \texttt{lni} uses the layout for articles in LNI. This document describes the usage of the \LaTeX-Class and gives some examples. This abstract should give a short overview about your work and can have between 70 and 150 words. The formatting will be done automatically within the abstract area.
\end{abstract}
\begin{keywords}
\end{keywords}

Investigate the literature from psychology, medicine and pattern recognition
for face beauty scoring methods

\section{Introduction}
In the recent years, artificially generated media (images, video, sound, etc.) has seen a noticable increase in production. This can in large part be attributed to the steady grow of better computers as well as more stuff.

One area that has been getting a lot of notice is AI-generated images. This can be seen in multiple big software solutions such as [Names i don't remember right now]. With this rise, it has become more important to enable software to have some idea of the likelyhood of an image being a fake or real.

As seen in ref[humans more beutiful huehue] humans tend to score higher on attractivness with more symmetric faces. This is further investigated in ref[] that comes to some conclusion.

With the enormous amounts of images being produced by GAN's and published on the interent, multiple sources worry that this could lead to a detereoration in results using AI, as the generated content might be slightly misleading.

To mitigate this, this survey will instead look at handcrafcted approaches that take spring from research regarding what parameters humans value other humans attractivness based on.

First a taxonomy is introduced expanding on the already established ISO[...] for describing compute-related tasks. Then an implementation of a current solution for a hand-crafted approach will be used. Lastly a survey of current handcrafted solutions will be shown, comparing several current solutions in there established benefits, speeds and more.

Lastly a comparison between the chosen implementation and surveyed implementation will be reviewed pondering on the benefits of the chosen solution as well as its weaknesses.

\section{Background}
This section gives an overview of background information regarding. Section \ref{nomenclature} explains relevant concepts and their background for stuff not covered by the 
\subsection{Nomenclature}\label{nomenclature}

\section{Methodology}
\section{Litterature Survey}
\section{Experimental evaluation}
From the handcrafted approaches it was determined, that \cite[]{j_iyer_machine_2021} implemented a good hybrid between hand-crafted features and using machine-learning to heighten the applicability.
The implementation of the approach was derived from instructions in their paper, although some things were not clear, and were therefor guesstimated.

The solution uses the following pipeline:
\begin{enumerate}
    \item Querry image
    \item Extract features from this image
        \subitem Facial landmarks
        \subitem Texture
        \subitem Color
        \subitem Shape
    \item Gather extracted features in one vector
    \item Use feature-vector as input to different machine-learning-models
\end{enumerate}

All features are handcrafted and based on solid research into what constitutes a pretty face seen from a human standpoint.

\subsection{Extracting features}
This section will discuss the different features extracted, how they are calculated and why they are used.

\subsubsection{Facial landmarks}

\subsubsection{Texture}

\subsubsection{Color}

\subsubsection{Shape}

\subsection{Models}

\subsection{Results}

\section{Discussion}

\bibliography{biometrics}

\end{document}
